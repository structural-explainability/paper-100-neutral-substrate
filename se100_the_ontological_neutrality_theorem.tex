\PassOptionsToPackage{colorlinks=true,allcolors=black}{hyperref}
% REQ-FILE: The above line, \PassOptionsToPackage{} must be very first line.

\documentclass[11pt]{article}

% ============================================================
% Annotation visibility toggle
% Build annotated PDF by defining \ANNOTATED (e.g., via latexmk flag).
% Requires xcolor loaded first, then use \textcolor and \color commands.
% ============================================================
\newif\ifShowAnnotations
\ifdefined\ANNOTATED
  \ShowAnnotationstrue
\else
  \ShowAnnotationsfalse
\fi

% Encoding and font setup for cross-platform reproducibility
\usepackage[T1]{fontenc}
\usepackage{lmodern}

% Improve readability for dense theoretical material
\usepackage{setspace}
\onehalfspacing

% Mathematical notation and table formatting for formal semantics
\usepackage{amsmath, amssymb}
\usepackage{array}
\usepackage[table]{xcolor}
\usepackage{colortbl}
\usepackage{booktabs}
\usepackage{graphicx}

% Theorem environments (best-practice, widely standard)
\usepackage{amsthm}

% ------------------------------------------------------------
% Numbering convention:
% - All theorem-like statements share one counter, numbered by section.
% - Definition/Example/Remark/Note also share the same counter family.
%   (This is the most common convention in math/logic/category theory writing.)
% ------------------------------------------------------------

% Italic body text: results
\theoremstyle{plain}
\newtheorem{theorem}{Theorem}[section]
\newtheorem{lemma}[theorem]{Lemma}
\newtheorem{proposition}[theorem]{Proposition}
\newtheorem{corollary}[theorem]{Corollary}
\newtheorem{conjecture}[theorem]{Conjecture}

% Claim varies; most standard is to share the theorem counter
\newtheorem{claim}[theorem]{Claim}

% Upright body text: definitions and examples
\theoremstyle{definition}
\newtheorem{definition}[theorem]{Definition}
\newtheorem{example}[theorem]{Example}

% Upright body text: remarks/notes
\theoremstyle{remark}
\newtheorem{remark}[theorem]{Remark}
\newtheorem{note}[theorem]{Note}

% Optional: unnumbered versions for front-matter previews/callouts
% (This is also standard when you want a preview without numbering.)
\newtheorem*{theorem*}{Theorem}
\newtheorem*{lemma*}{Lemma}
\newtheorem*{proposition*}{Proposition}
\newtheorem*{corollary*}{Corollary}
\newtheorem*{conjecture*}{Conjecture}
\newtheorem*{claim*}{Claim}
\newtheorem*{definition*}{Definition}
\newtheorem*{example*}{Example}
\newtheorem*{remark*}{Remark}
\newtheorem*{note*}{Note}

% Bibliography management and hyperlink support
\usepackage{url}
\usepackage{natbib}
% load hyperref before bookmark to avoid warnings
\usepackage{hyperref}
\usepackage{bookmark}

% Support multiple author affiliations
\usepackage{authblk}

% Improve typographic quality and line breaking
\usepackage{microtype}

% Categorical and commutative diagrams
\usepackage{tikz}
\usepackage{tikz-cd}
\usetikzlibrary{arrows.meta,calc,fit,positioning,shapes.geometric}

% Callout boxes for examples and emphasis
\usepackage[most]{tcolorbox}

% Keywords macro for structured abstract metadata
\providecommand{\keywords}[1]{\textbf{Keywords: } #1}

% Notation for Raw vs Canonical categories in EP/CEE
\newcommand{\Raw}{\mathsf{Raw}}
\newcommand{\Canon}{\mathsf{Canon}}

% Reusable figure callout box for visual emphasis
\newcommand{\FigureCallout}[2]{%
  \begin{tcolorbox}[
    colback=gray!5,
    colframe=black!40,
    title={#1},
    fonttitle=\bfseries,
    arc=3pt,
    boxrule=0.5pt,
    width=\linewidth,
    enhanced,
    breakable
  ]
  #2
  \end{tcolorbox}
}

% ============================================================
% Annotations (REQ / WHY / OBS) scan-friendly
% ============================================================
\newcommand{\AnnColor}{gray} % xcolor color name
\newlength{\AnnHangIndent}
\setlength{\AnnHangIndent}{1.8em} % hanging indent amount

% Single annotation line/block:
% - small gray text
% - bold label REQ(P0) then content
% - hanging indent for scanning
\newcommand{\AnnInline}[3]{%
  \ifShowAnnotations
    \par\begingroup
      \footnotesize
      \color{\AnnColor}%
      \noindent
      \hangindent=\AnnHangIndent
      \hangafter=1
      \textbf{#1(#2):}~#3\par
    \endgroup
    \vspace{0.15\baselineskip}%
  \fi
}
\newcommand{\REQ}[2]{\AnnInline{REQ}{#1}{#2}}
\newcommand{\WHY}[2]{\AnnInline{WHY}{#1}{#2}}
\newcommand{\OBS}[2]{\AnnInline{OBS}{#1}{#2}}

% Paragraph spacing prioritizes readability over traditional indentation
\setlength{\parskip}{0.75em}
\setlength{\parindent}{0em}

% Compact list formatting for dense conceptual content
\usepackage{enumitem}
\setlist[itemize]{itemsep=0.2em, topsep=0.2em, parsep=0em, partopsep=0em}

% Defensive re-definition ensures commands exist if loaded earlier
\providecommand{\Raw}{\mathrm{Raw}}
\providecommand{\Canon}{\mathrm{Canon}}

% ============================================================
\title{The Ontological Neutrality Theorem:\\
Why Neutral Ontological Substrates Must Be \\
Pre-Causal and Pre-Normative}
% ============================================================

\author[1,2]{Denise M. Case}
\affil[1]{Northwest Missouri State University, Computer Science and Information Systems, Maryville, MO, USA}
\affil[2]{Civic Interconnect, Ely, MN, USA}

\date{\today}

\begin{document}

\maketitle
\vspace{-1em}

% Explicitly mark preprint status
\begin{tcolorbox}[colback=gray!5, colframe=black!20, boxrule=0.3pt]
  \textbf{Preprint Notice.}
  This preprint has not undergone peer review.
  It is shared to support community discussion, transparency research, and early technical evaluation.
\end{tcolorbox}

\begin{abstract}
  Modern data systems must support accountability
  across persistent legal, political, and analytic disagreement.
  This requirement imposes strict constraints
  on the design of any ontology intended to function as a shared substrate.
  We establish an impossibility result for ontological neutrality:
  neutrality, understood as interpretive non-commitment and stability
  under incompatible extensions, is incompatible with the inclusion of
  causal or normative commitments at the foundational layer.
  Any ontology that asserts causal or deontic conclusions
  as ontological facts cannot serve as a neutral substrate across divergent
  frameworks without revision or contradiction.
  It follows that neutral ontological substrates
  must be pre-causal and pre-normative,
  representing entities, together with identity and persistence conditions,
  while externalizing interpretation, evaluation, and explanation.
  This paper does not propose a specific ontology or protocol;
  rather, it establishes the necessary design constraints
  for any system intended to maintain a shared,
  stable representation of reality across conflicting interpretive frameworks.
\end{abstract}

% =============================================================================
% Theorem Preview 
% =============================================================================

\bigskip
\noindent\textbf{Statement of Result (Preview).}
\emph{Let $\mathcal{O}$ be an ontology intended to function
  as a neutral substrate across diverse interpretive frameworks.
  Then $\mathcal{O}$ satisfies the requirements of neutrality
  if and only if its foundational Level of Abstraction
  excludes causal and normative primitives.}

\medskip
\noindent
Informally: Neutrality is not achieved by
finding uncontested causes,
but by externalizing the category of causation entirely.
To be a substrate for disagreement, an ontology must be
pre-causal and pre-normative.

\medskip
\noindent
The formal statement and proof appear in Section~\ref{sec:impossibility}.

\keywords{
  Formal ontology;
  ontological neutrality;
  accountability;
  neutrality constraints;
  extension stability;
  deontic and causal commitment
}

\ifShowAnnotations
  \clearpage
  \input{sections/00_contract}
  \clearpage
\fi

\input{sections/01_intro}           % Introduction: Accountability, Disagreement, and Neutral Substrates
\input{sections/02_related}         % Related Work
\input{sections/03_requirements}    % Formal Requirements for Ontological Neutrality
\input{sections/04_impossibility}   % The Impossibility of Causal and Normative Substrates
\input{sections/05_implications}    % Implications for Ontology Design 
\input{sections/06_conclusion}      % Conclusion


\section*{Statements and Declarations}

\subsection*{Author Contributions}
The author was solely responsible for the conception, analysis, and writing of this manuscript.

\subsection*{Declaration of Conflicting Interest}
The author declares no potential conflicts of interest
with respect to the research, authorship,
and/or publication of this article.

\subsection*{Funding}
This research received no external funding.

\subsection*{Data Availability}
No datasets were generated or analyzed
during the current study.

\subsection*{Ethical Approval}
Not applicable.

\subsection*{Consent to Participate}
Not applicable.

\subsection*{Consent for Publication}
Not applicable.

\appendix
\clearpage

\input{sections/99_appendixG}   % Glossary

\ifShowAnnotations
  \clearpage
  \input{sections/99_appendixZ}   % Trace (internal)
\fi

\bibliographystyle{plainnat}
\bibliography{bibliography/bibliography}
\end{document}